% Gemini theme
% See: https://rev.cs.uchicago.edu/k4rtik/gemini-uccs
% A fork of https://github.com/anishathalye/gemini

\documentclass[final]{beamer}

% ====================
% Packages
% ====================

\usepackage[size=custom,width=36,height=48,scale=0.5]{beamerposter}
\usetheme{gemini}
\usecolortheme{stanford}
\usepackage{graphicx}
\usepackage{booktabs}
\usepackage{tikz}
\usetikzlibrary{shapes, arrows, positioning}
\usepackage{pgfplots}
\pgfplotsset{compat=1.17}
\usepackage{fontspec}
\usepackage{exscale}
\usepackage{tabularx}
\usepackage{colortbl} % For \rowcolor
\usepackage{xcolor}   % For color definitions
\usepackage{tcolorbox}
\usepackage[backend=biber,style=numeric,maxnames=3,minnames=1]{biblatex}
\addbibresource{poster.bib} % Adjust the filename as needed
\usepackage{amssymb}

\setbeamercolor{headline}{bg=cardinalred,fg=white}
\setbeamercolor{footline}{bg=footergray,fg=darkcardinal}

% ====================
% Lengths
% ====================

% If you have N columns, choose \sepwidth and \colwidth such that
% (N+1)*\sepwidth + N*\colwidth = \paperwidth
\newlength{\sepwidth}
\newlength{\colwidth}
\setlength{\sepwidth}{0.025\paperwidth}
\setlength{\colwidth}{0.3\paperwidth}

\newcommand{\separatorcolumn}{\begin{column}{\sepwidth}\end{column}}


\title{How Did a Strength-Based Video-Coaching Intervention Alter Parental Neurocognitive Mechanisms? Evidence From RCT \newline Studies in Low-Income, High-Adversity Contexts}


\author{\fontsize{16}{18}\selectfont Howard Chiu \inst{1} \and Nicole Giuliani \inst{2} \and Sihong Liu \inst{3} \and Phil Fisher \inst{3} }

\institute[shortinst]{%
  \inst{1} Graduate School of Education, Stanford University, Stanford, CA, USA \\ 
  \inst{2} College of Education, University of Oregon, Eugene, OR, USA \\ 
  \inst{3} Stanford Center on Early Childhood, Stanford University, Stanford, CA, USA
}

% ====================
% Footer (optional)
% ====================

\footercontent{
  \href{chiuhoward.github.io}{chiuhoward.github.io} \hfill
  Thank you to all the families that made this research possible! \hfill
  \href{mailto:howardchiu@stanford.edu}{howardchiu@stanford.edu}}
% (can be left out to remove footer)

% ====================
% Logo (optional)
% ====================

% use this to include logos on the left and/or right side of the header:
% \logoright{\includegraphics[height=7cm]{logos/cs-logo-maroon.png}}
% \logoleft{\includegraphics[height=7cm]{logos/cs-logo-maroon.png}}

% ====================
% Body
% ====================

\begin{document}

% This adds the Logos on the top left and top right
\addtobeamertemplate{headline}{}
{
  \begin{tikzpicture}[remember picture,overlay]
    \node [anchor=north east, inner sep=3cm] at ([xshift=2.25cm,yshift=-2.2cm]current page.north east)
    {\includegraphics[width=6.25cm]{stanford_logos/GSE-hor-logo-white.png}};
  \end{tikzpicture}

  \begin{tikzpicture}[remember picture,overlay]
    \node [anchor=north east, inner sep=3cm] at ([xshift=19.5cm,yshift=-1.35cm]current page.north east)
    {\includegraphics[width=38.0cm]{stanford_logos/stanford-line2-10-white.png}};
  \end{tikzpicture}
}

\begin{frame}[t]
\begin{columns}[t]
\separatorcolumn

\begin{column}{\colwidth}
  \vspace{-15pt}
  \begin{block}{Introduction}
    Filler

    \begin{columns}[t] % Ensure top alignment
      % Left column with the TikZ picture
      \begin{column}{0.45\textwidth}
        \vspace{5pt}
        \begin{tikzpicture}
          % Actual age bar (longer bar)
          \filldraw[fill=blue!30, draw=none] (0,0) rectangle (6,1);
          
          % Predicted age bar (shorter bar)
          \filldraw[fill=orange!60, draw=none] (0,0) rectangle (4,1);
          
          % Text annotations
          \node at (2,0.5) {Predicted Age};
          \node at (3,1.3) {Actual Age};
          
          % Arrow showing the brain age gap
          \draw[<->,thick] (4.2, 0.7) -- (5.8, 0.7) node[midway, below] {Gap};
          
          % Additional arrows
          \draw[<-,thick] (0.2, 1.3) -- (1.8, 1.3);
          \draw[->,thick] (4.2, 1.3) -- (5.8, 1.3);
        \end{tikzpicture}
      \end{column}
    
      % Right column with the colored text box
      \begin{column}{0.5\textwidth}
        \vspace{-20pt} % Adjust this value to vertically align
        \begin{tcolorbox}[colback=blue!10!white, colframe=blue!40!black, width=\linewidth]
          {\RaggedRight{Can we conceptualize a \\ positive brain age gap as a distal summary statistic for \textbf{poorer} infant health?}}
        \end{tcolorbox}
      \end{column}
    \end{columns}
  \end{block}

  \vspace{-5pt}

  \begin{block}{Methods}
    \begin{tikzpicture}[node distance=3cm, auto]
        % Define block styles
        \tikzstyle{block} = [rectangle, draw, fill=cardinalred, 
        text width=6em, text centered, rounded corners, minimum height=3em, font=\small, text=white]
        \tikzstyle{line} = [draw, -latex']

        % Nodes
        \node [block] (dmridata) {Raw diffusion MRI data};
        \node [block, right of=dmridata, xshift = 0.25cm] (qsiprep) {Preprocessing in QSIprep$^1$};
        \node [block, right of=qsiprep, xshift = 0.25cm] (pyAFQ) {Reconstruction \\ and \\ segmentation of whole brain tractogram in pyAFQ$^3$};
        \node [block, right of=pyAFQ, xshift = 0.25cm] (AFQInsight) {Models trained on tract profiles and target features in AFQ-Insight$^5$};

        % Lines
        \path [line] (dmridata) -- (qsiprep);
        \path [line] (qsiprep) -- (pyAFQ);
        \path [line] (pyAFQ) -- (AFQInsight);
    \end{tikzpicture}
    
    \vspace{-10pt}
    
    \begin{itemize}
        \item 3 Models
        \begin{itemize}
            \normalsize % Set the font size to normal for sub-items
            \item Principal components regression (PCR) - No regularization
            \item Lasso principal components regression (PCR-Lasso) - L1 regularization enforces sparsity
            \item Bundle-wise principal components regression with sparse group lasso penalties (PCR-SGL) - L1 and L2 regularization
        \end{itemize}
        \vspace{10pt}
        \item 3 Targets
        \begin{itemize}
            \normalsize % Set the font size to normal for sub-items
            \item GA (age since conception)
            \item Chronological age (CA; age since delivery)
            \item Post-menstrual age (PMA) at scan (sum of GA + CA)
        \end{itemize}
    \end{itemize}

    \vspace{10pt}

        \begin{tikzpicture}
        % Define colors and dimensions
        \def\barHeight{0.8}         % Height of each bar
        \def\barLengthTotal{5}       % Total length of each bar (representing 100% of data)
        \def\testBarLength{1}        % Length of test set (20%)
        \def\trainBarLength{4}       % Length of training set (80%)
        \def\hOffset{0.7}            % Horizontal displacement between splits
        \def\vOffset{0.4}            % Vertical displacement between splits
        \def\nSplits{5}              % Number of splits
        \def\offset{0.4}             % Text offset for better centering
    
        % Loop through and draw the 5 displaced bars for each split
        \foreach \i in {1,...,\nSplits} {
            % Calculate vertical and horizontal shift for each bar
            \pgfmathsetmacro{\xShift}{(\i-1)*\hOffset}
            \pgfmathsetmacro{\yShift}{(\i-1)*\vOffset}
            \pgfmathsetmacro{\testStart}{(\i-1)*\testBarLength} % Test set shift for highlighting
            
            % Total bar length
            \draw[fill=none, draw=black] 
            (\xShift, \yShift) rectangle (\xShift+\barLengthTotal, \yShift+\barHeight);
    
            % Highlight the test set section (20% of data)
            \filldraw[fill=blue!40, draw=black] 
            (\xShift+\testStart, \yShift) rectangle (\xShift+\testStart+\testBarLength, \yShift+\barHeight);
            
            % Fill the remaining part of the bar for the training set
            \filldraw[fill=orange!60, draw=black] 
            (\xShift, \yShift) rectangle (\xShift+\testStart, \yShift+\barHeight);
            \filldraw[fill=orange!60, draw=black] 
            (\xShift+\testStart+\testBarLength, \yShift) rectangle (\xShift+\barLengthTotal, \yShift+\barHeight);
        }

        % Text annotations with smaller font
        \node[font=\small] at (\xShift+\barLengthTotal, \yShift+\barHeight+1.25) {Train (80\% of data)};
        \node[font=\small] at (\xShift+\barLengthTotal+2.3, \yShift+\barHeight+1.25) {Test};
            
        % Repetition arrow
        \draw[->, thick] (\xShift+\barLengthTotal+2.5, \yShift+\barHeight) -- ++(5.5,0)
        node[midway, above] {Repeat nested cross-validation};
    
    \end{tikzpicture}


    \vspace{10pt}
    
    \begin{tikzpicture}[node distance=3cm, auto]
        % Define block styles
        \tikzstyle{block} = [rectangle, draw, fill=paloaltogreen, 
        text width=6em, text centered, rounded corners, minimum height=3em, font=\small, text=white]
        \tikzstyle{line} = [draw, -latex']

        % Nodes
        \node [block] (rawdata) {Raw data \\ available with T1, pe0 and pe1 \\ (\textit{n} = 359)};
        \node [block, right of=rawdata, xshift = 0.25cm] (qsiprep-pe1) {Completed QSIprep \\ (\textit{n} = 316)};
        \node [block, right of=qsiprep-pe1, xshift = 0.25cm] (pyAFQ-pe1) {Completed pyAFQ \\ (\textit{n} = 287)};
        \node [block, right of=pyAFQ-pe1, xshift = 0.25cm] (AFQInsight-pe1) {WM features included in brain age model after passing QC \\ (\textit{n} = 184)};

        % Lines
        \path [line] (rawdata) -- (qsiprep-pe1);
        \path [line] (qsiprep-pe1) -- (pyAFQ-pe1);
        \path [line] (pyAFQ-pe1) -- (AFQInsight-pe1);
    \end{tikzpicture}

    \vspace{-2pt}
    Their health acuity was evaluated based on a sum of binary indicators for 4 major comorbidities of prematurity (mean = 0.69, range: 0 -- 4):
    \vspace{-5pt}
    \begin{itemize}
    \item Sepsis 
    \item Necrotizing enterocolitis 
    \item Intraventricular hemorrhages (IVH) of grade 1 and above 
    \item Bronchopulmonary dysplasia of grade 2 and above 
    \end{itemize}
      
  \end{block}

\end{column}

\separatorcolumn

\begin{column}{\colwidth}
  \vspace{-15pt}
  \begin{block}{Participants (\textit{n} = 184; males = 104) were infants born at \textless 32 weeks GA. CA and GA were highly negatively correlated (\textit{r} = -0.85).}
    \vspace{-5pt}
    \begin{figure}[ht]
      \centering
      \includegraphics[clip, width=0.8\textwidth]{age_at_scan.jpg}
      \label{fig:age_at_scan}
    \end{figure}
  \end{block}
  \vspace{-20pt}
  
  \begin{block}{Right anterior thalamic radiation (ATR) microstructure contributes the most to the prediction of brain age in this sample.}
    Filler

    \vspace{-10pt}
    
    \begin{figure}
      \centering
      \includegraphics[clip, width=0.8\textwidth]{sgl_both.jpg}
    \end{figure}
    \vspace{-20pt}  % Adjust the value to control the space

    
    \begin{columns}[t] % Ensure top alignment
        \begin{column}{0.53\textwidth}
            \vspace{-10pt}
            \begin{figure} % sub-17055, sub-17044
            \centering  \includegraphics[clip,width=\textwidth]{brain_2.png}
            \end{figure}
      \end{column}
    
      % Right column with the colored text box
      \begin{column}{0.46\textwidth}
        \vspace{-10pt}
        \begin{tcolorbox}[colback=gray!50, colframe=cardinalred, width=\linewidth]
          {\fontsize{9.5}{10}\selectfont
          \begin{itemize}
            \item \textcolor[rgb]{1.0, 1.0, 0.4}{\rule{0.8cm}{0.3cm}} Left Arcuate (ARC)
            \item \textcolor[rgb]{0.2, 0.4, 0.6}{\rule{0.8cm}{0.3cm}} Left Anterior Thalamic Radiation (ATR) 
            \item \textcolor[rgb]{1.0, 0.4, 0.0}{\rule{0.8cm}{0.3cm}} Left Superior Longitudinal Fasciculus (SLF) 
            \item \textcolor[rgb]{0.0, 0.6, 0.2}{\rule{0.8cm}{0.3cm}} Left Uncinate (UNC)
          \end{itemize}
          Left hemisphere tracts plotted for illustrative purposes. Tract endpoints have been clipped in pyAFQ.}
        \end{tcolorbox}
      \end{column}
    \end{columns}
    
  \end{block}

\end{column}

\separatorcolumn

\begin{column}{\colwidth}
  \vspace{-15pt}
  \begin{block}{Infant WM features explain 21\% of variance in PMA at scan. Including both FA and MD increases test $R^2$.}

    Filler. 
    \begin{table}[ht]
      \centering
      \fontsize{9}{11}\selectfont
      \begin{tabularx}{\textwidth}{l c c c c c c}
        \toprule
        \textbf{Model} & \textbf{Target} & \textbf{Train $R^2$} & \textbf{Test $R^2$} & \textbf{Train MAE} &
        \textbf{Test MAE} \\
        \midrule
         PCR & PMA at scan & 0.559 & 0.198 & 8.14 & 9.84 \\
        \midrule
        \rowcolor{yellow!50} PCR-Lasso & PMA at scan & 0.491 & 0.208 & 6.99 & 7.67 \\
        PCR-Lasso (FA only) & PMA at scan & 0.428 & 0.144 & 6.75 & 7.43 \\
        PCR-Lasso (MD only) & PMA at scan & 0.180 & 0.047 & 8.30 & 8.71 \\
        \midrule
        \rowcolor{yellow!50} PCR-SGL & PMA at scan & 0.264 & 0.104 & 8.05 & 8.19 \\
        PCR-SGL (FA only) & PMA at scan & 0.220 & 0.054 & 8.14 & 8.25 \\
        PCR-SGL (MD only) & PMA at scan & 0.170 & 0.041 & 8.47 & 8.82 \\
        \bottomrule
      \end{tabularx}
    \end{table}
  \end{block}

  \vspace{-15pt}
  
  \begin{block}{Brain age gap explains additional 2\% of variance in health acuity above and beyond CA and GA. \\ Variance explained by brain age gap is greater than any WM feature alone (mean tract FA or MD).}
  
    \vspace{-25pt}

    \[    
    \textbf{Model 2:} \quad \mathrm{health\_acuity} \sim \mathrm{1} + \mathrm{CA} + \mathrm{GA} + \mathrm{brain\_age\_gap}
    \]

    \vspace{-15pt}

       \begin{table}[htbp]
        \centering
        \fontsize{12}{14}\selectfont
        \begin{tabularx}{\textwidth}{X c c c}
        \toprule
        \textbf{Variables} & \textbf{Model 1} & \textbf{Model 2} & \textbf{Model 3} \\
        \midrule
        Intercept & -0.280 & 2.93 & \checkmark \\
        CA & 0.018*** & 0.005 & \checkmark \\
        GA & 0.001 & -0.013 & \checkmark \\
        PMA at Scan Gap & - & 0.019** & - \\
        Mean Tract FA or MD & - & - & \checkmark \\
        \midrule
        \textbf{Likelihood Ratio (LR)} & - & 6.91** & \\
        \bottomrule
        \end{tabularx}
      \end{table}
      \vspace{-10pt}
     \caption{\fontsize{10}{12}\selectfont Asterisks denote level of significance: *** $\textit{p} < .001$, ** $\textit{p} < .01$, * $\textit{p} < .05$}
    \end{block}
    
    \vspace{-10pt}  % Adjust the value to control the space

   \begin{tcolorbox}[colback=blue!10!white, colframe=blue!40!black, width=\linewidth, title=\centering \textbf{Conclusion}]
    The brain age gap \textbf{shows promise} as a summary statistic for infant health, above and beyond information that can be gleaned from age and individual measures of WM microstructure.
    \end{tcolorbox}
        
    \begin{block}{References}
        \nocite{*}
        \renewcommand{\bibfont}{\fontsize{6}{7}\selectfont} % Use \footnotesize, \small, \scriptsize, or \tiny as needed
        \printbibliography
    \end{block}
  
\end{column}

\separatorcolumn
\end{columns}
\end{frame}

\end{document}